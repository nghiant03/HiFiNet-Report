\begin{abstract}
  Wireless Sensor Networks (WSN) are integral to various monitoring applications, but their deployment in harsh environments makes them susceptible to faults, which can compromise data integrity and system reliability. Traditional fault detection methods often struggle to effectively balance accuracy and energy consumption, and they may not fully leverage the complex spatio-temporal correlations inherent in WSN data. In this paper, we introduces HiFiNet, a novel hierarchical fault identification framework that addresses these challenges through a two-stage process. Initially, an edge classifier using a Long Short-Term Memory (LSTM) based stacked autoencoder performs temporal feature extraction and preliminary fault screening on individual sensor nodes. Subsequently, a Graph Attention Network (GAT) aggregates information from neighboring nodes, refining the fault diagnosis by considering the broader spatial context. This hierarchical approach, combining edge classification with graph aggregation, allows for a comprehensive analysis that captures both local temporal patterns and network-wide spatial dependencies. To validate our approach, we created synthetic WSN datasets by injecting various fault types into real-world environmental data from the Intel Lab Dataset and NASA's MERRA-2 reanalysis data. Experimental results demonstrate that HiFiNet significantly outperforms existing methods in terms of accuracy, F1-score, and precision, showcasing its robustness and effectiveness in identifying diverse fault types. Furthermore, the framework's design allows for a tunable trade-off between diagnostic performance and energy efficiency, making it adaptable to different operational requirements.
\end{abstract}
