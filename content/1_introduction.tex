Wireless Sensor Networks (WSN) are widely used in a variety of fields such as healthcare, logistics, military, and environment monitoring. These networks, comprised of spatially distributed sensor nodes, support real-time data acquisition and remote monitoring, offering enhanced decision-making and rapid response capabilities in critical scenarios like structural health monitoring and environmental surveillance \cite{Yick2008, Chai2020, Ullo2020}. WSNs are often deployed in harsh and challenging environments where traditional wired solutions are impractical or impossible, ranging from inaccessible terrains, such as in the monitoring of volcanoes or areas with sliding mud, to industrial settings with extreme temperatures, humidity, or vibrations \cite{Gungor2009, Prasad2023}.

This versatility does not come without drawbacks however. The individual sensor nodes are often resource-constrained and operate in unattended conditions, making them highly prone to malfunctions and data corruption. These undetected errors can have severe consequences by generating incorrect predictions, which compromise the reliability of monitoring and control systems and may result in system-wide damage. These reasons explain why a robust fault detection system is crucial. Faults in WSN data are typically categorized based on their temporal behavior (time-based) or their impact on sensor readings (characteristic-based) \cite{Baljak2013, Adday2022}. Figure~\ref{fig:types} depicts the fault types from both perspectives. From a time-based perspective, faults can be classified as soft permanent, intermittent, and transient faults \cite{Prasad2023}. Characteristic-based faults, which describe how the data values themselves are corrupted (e.g., becoming fixed, shifted, or exhibiting other anomalous patterns), also exhibit diverse types according to the literature \cite{Shi2024, Saeed2021, Ni2009, Hasan2024}. While both categorizations are valid, this study concentrates on characteristic-based faults because they provide more direct, actionable insights into the potential root cause. The specific characteristic-based fault types considered in this study will be detailed in our fault taxonomy (see Section~\ref{subsec:types}). Understanding these fault types informs the design of the detection algorithms, which we categorize next.

\begin{figure}
  \centering
  \includegraphics[width=0.8\linewidth]{images/fault_taxonomy.png}
  \caption{WSN Fault Taxonomy}
  \label{fig:types}
\end{figure}

To address the problem of identifying characteristic-based fault types, researchers have employed methods such as model-based approaches, data-driven approaches and hybrid information-based methods \cite{Shi2024}. The most common approach is a model-based algorithm, which utilize mathematical and statistical principles to model each fault type \cite{Panda2014, Ahmad2024}. The data-driven approach on the other hand uses the analysis of data samples obtained to build a model for fault classification \cite{Saeed2021, Prasad2023}. Hybrid information-based methods use both human knowledge and data through a combination of different methods \cite{Sun2023, Shi2024}. Prior work either focuses on temporal patterns at individual nodes or spatial relations across neighbors, but fails to integrate both effectively. However, combining both approaches is crucial, since relying heavily on neighbor-based detection drains nodes’ energy and reduces their lifespans, while using only self-detection methods leads to substantial bias. Therefore, a hierarchical architecture that can effectively model both local temporal dynamics and broader spatial dependencies offers a balanced solution between accuracy and energy consumption. Furthermore, given the complexity of WSN data and the potential for unknown fault patterns, data-driven approaches that can learn from the data themselves are particularly promising. In this work, we investigate the efficacy of a data-driven hierarchical method for WSNs fault diagnosis. The major contribution of this paper are as follows:

\begin{itemize}
  \item Introduce a novel Iterative Graph Network (IGN) that integrates Graph Attention Convolution with a custom Confidence Modulator(CM), dynamically refining node representations through iterative confidence propagation.
  \item Propose HiFiNet, a hierarchical network. HiFiNet first employs a Long Short-Term Memory based stacked autoencoder (LSTM-SAE) at the edge for temporal feature extraction and initial fault screening, followed by IGN for aggregating neighborhood information and refining fault diagnosis.
  \item Simulate real-world fault scenarios by generating synthetic WSN datasets that combines the Intel Lab Dataset measurements \cite{Intel2004} with environmental measurements drawn from NASA’s MERRA‑2 reanalysis data \cite{GMAO2015}.
  \item Evaluate performance including various metrics such as: accuracy, F1-score, precision on the aforementioned datasets, demonstrate improvement against methods in the literature.
\end{itemize}
